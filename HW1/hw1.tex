\documentclass{article}
%\documentclass{beamer}
\usepackage{graphicx,fancyhdr,amsmath,amssymb,amsthm,subfig,url,hyperref}
\usepackage[margin=1in]{geometry}
\usepackage{subfig}
\usepackage[utf8]{inputenc}
\usepackage{amsmath}
\DeclareUnicodeCharacter{2212}{-}
%\DeclareUnicodeCharacter{2265}{≥}
% \DeclareUnicodeCharacter{2264}{$\leq$}
% \DeclareUnicodeCharacter{2265}{$\beq$}


%----------------------- Macros and Definitions --------------------------

%%% FILL THIS OUT
\newcommand{\SecondAuther}{Mohammad Parsa Dini}
\newcommand{\exerciseset}{Homework Set 1 (Sol)}
%%% END



\renewcommand{\theenumi}{\bf \Alph{enumi}}

%\theoremstyle{plain}
%\newtheorem{theorem}{Theorem}
%\newtheorem{lemma}[theorem]{Lemma}

\fancypagestyle{plain}{}
\pagestyle{fancy}
\fancyhf{}
\fancyhead[RO,LE]{\sffamily\bfseries\large Sharif University of Technology}
\fancyhead[LO,RE]{\sffamily\bfseries\large EE 25-120: Deep Generative Models}
\fancyfoot[LO,RE]{\sffamily\bfseries\large Problem Set 1}
\fancyfoot[RO,LE]{\sffamily\bfseries\thepage}
\renewcommand{\headrulewidth}{1pt}
\renewcommand{\footrulewidth}{1pt}

\graphicspath{{figures/}}

%-------------------------------- Title ----------------------------------

\title{Deep Generative Models \par \exerciseset}
\author{\SecondAuther }

%--------------------------------- Text ----------------------------------

\begin{document}
\maketitle


\section*{Problem 1}
\begin{enumerate}
    \item  We want to do MLE parameter estimation for a linear autoregressive model $AR(p)$. The  $AR(p)$ is described by: $\forall t: y_t = \sum_{i=1}^p \phi_{i} y_{t-i} + \epsilon_t$ where $\epsilon_t$'s are identically and independently from $\mathcal{N}(0, \sigma^2)$ distribution. We assume that we $y_0$ is known and fixed, so $p(y_0)=1$. By chain  rule we have:
    \begin{equation*}
        p(y_0, ..., y_n)= \prod_{t=1}^n p(y_t | y_{t-1}, ..., y_1) = \prod_{t=p+1}^n p(y_t | y_{t-1}, ..., y_{t-p}) =\prod_{t=1}^{n} \mathcal{N}(\sum_{i=1}^p \phi_{i} y_{t-i} , \sigma^2)
    \end{equation*}    
    This is due to the fact that $\forall k: \mathbb{E} [y_k | y_{k-1} , ... , y_{k-p}] =  \sum_{i=1}^p \phi_{i} y_{k-i}$ and $\text{Var}(y_k | y_{k-1} , ... , y_{k-p}) = \sigma^2$. Therefore:
    \begin{equation*}
        p(y_0, ..., y_n)= \prod_{t=p+1}^n (\frac{1}{2\pi \sigma^2})^{\frac{1}{2}} \exp{(-\frac{1}{2 \sigma^2} (y_t - \sum_{i=1}^p \phi_{i} y_{t-i})^2 )}
    \end{equation*}    
    Now we derive the Log-Likelihood function which is(let $\theta = (\sigma, \{\phi_{i}\}_{i=1}^{p}$):
    \begin{equation*}
        \mathcall{L}(\theta) = -\log p(y_0, ..., y_n) = \frac{n-p}{2} \log(2\pi \sigma^2)  + \frac{1}{2 \sigma^2}\sum_{t=p+1}^{n}  (y_t - \sum_{i=1}^p \phi_{i} y_{t-i})^2 
    \end{equation*}  
    % ---------------------
    \item  Now in order to maximize the log-likelihood function, we differentiate $\mathcall{L}(\theta)$ with respect to $\sigma$ and $\phi_i$'s which  implies that:
    \begin{equation*}
        \frac{\partial \mathcall{L}(\theta)}{\partial \sigma} = (n-p) \frac{1}{\sigma} - \frac{1}{\sigma^3} \sum_{t=p+1}^{n}  (y_t - \sum_{i=1}^p \phi_{i} y_{t-i})^2 \Rightarrow \sigma = \sqrt{ \frac{1}{n-p} \sum_{t=p+1}^{n}  (y_t - \sum_{i=1}^p \phi_{i} y_{t-i})^2 }
    \end{equation*}
    \begin{equation*}
        \forall j \in \{1,2,..,p\}: \frac{\partial \mathcall{L}(\theta)}{\partial \phi_j} = \frac{ y_{t-j}}{\sigma^2} \sum_{t=p+1}^{n}  (y_t - \sum_{i=1}^p \phi_{i} y_{t-i}) = 0 \Rightarrow (y_{t-j} = 0) \lor ( \sum_{t=p+1}^{n}  (y_t - \sum_{i=1}^p \phi_{i} y_{t-i}) = 0)
    \end{equation*} 
    which clearly can be solved for $\phi_i$'s by having $\{y_0,..,y_n\}$ values.
\end{enumerate}
%-----------------------------------------------------------------------------
%\begin{enumerate} % for enumerating A,B,C
%    \item 
%\end{enumerate}
%-----------------------------------------------------------------------------
\section*{Problem 2}
\begin{enumerate} % for enumerating A,B,C
    \item In an autoregressive model we are given a sequence of continuous random variables  $x = (X_1, ..,X_T)$, and each $x_t$ given  $x_1,..,x_{t-1}$ comes from $\mathcal{N}(f_\theta(x_1,...,x_{t-1}) , \sigma^2)$. We wish to get the likelihood function:
    \begin{equation*}
         \mathcall{L}(\theta) = \log p(x_1,...,x_n) = \log (p(x_1) \prod_{k=2}^T p(x_k|x_1,..,x_{k-1} ) ) = \log(\prod_{k=1}^T \mathcal{N}(f_\theta(x_1,...,x_{k-1}) , \sigma^2))
    \end{equation*} 
    \begin{equation*}
         \mathcall{L}(\theta) = \log \prod_{k=1}^T (\frac{1}{2\pi \sigma^2})^{\frac{1}{2}} exp{(-\frac{(x_k - f_\theta(x_1,...,x_{k-1}))^2}{2 \sigma^2})} =
         - \frac{T}{2} \log(2\pi \sigma^2) - \frac{1}{2\sigma^2} 
         \sum_{k=1}^T (x_k - f_\theta(x_1,...,x_{k-1}))^2
    \end{equation*}
    In order to maximize the likelihood function, one must minimize the term $\sum_{k=1}^T (x_k - f_\theta(x_1,...,x_{k-1}))^2$. In other words we have an $MSE$ loss function as $l(\theta) = \sum_{k=1}^T (x_k - f_\theta(x_1,...,x_{k-1})^2$. 
    Thus, in order to train the neural network $f\theta$, we must
    compute the gradient of the loss. Therefore, we get:
    \begin{equation*}
        \nabla_\theta \mathcall{L}(\theta) = \frac{-1}{\sigma^2} \sum_{k=1}^T  (x_k -f_\theta(x_1,...,x_{k-1})) 
        
    \end{equation*}
    So in order to find $\theta$, we can use an optimizer like SGD to update weights $\theta$ each step($\theta_{t+1} \rightarrow \eta \nabla_\theta \mathcall{L}(\theta_t) $.
    % ---------------------------------------------------------------------
    \item  Generating a new sequence $(x'_1, ..., x'_T)$ can't be processed in parallel due to sequential dependency. Since for each variable $x'_k$ (that  $ x'_k | x'_1 , ..., x'_{k-1} \sim \mathcal{N}(f_\theta(x'_1,...,x'_{k-1}) , \sigma^2))$), the previous variables play their part by setting the mean of the normal distribution: $\mathbb{E} x_t = f_\theta(x_1,...,x_{k-1})$. Thus, for generating each variable, all the variables prior to that one, must be generated first.
    % ---------------------------------------------------------------------
    \item Firstly, we take this as a lemma that for independent gaussian distributions $p_\theta \sim  \mathcal{N}(\mu_1, \sigma^2)$ and $q_\Phi \sim  \mathcal{N}(\mu_2, \sigma^2)$, then their KL-divergence will be $D_{KL}(p_\theta || q_\Phi) = \frac{(\mu_1 - \mu_2)^2}{2 \sigma^2} $.
    Using chain rule we obtain:
    \begin{equation*}
        D_{KL}(p_\theta || q_\Phi) = \mathbb{E}_{x \sim p_\theta}
        [\log (\frac{p_\theta(x)}{q_\Phi(x)}] = \mathbb{E}_{x \sim p_\theta} 
        [\log (\frac{p_\theta(x_1) \prod_{k=2}^T p_\theta(x_k|x_1,..,x_{k-1} ) ) }{q_\Phi(x_1) \prod_{k=2}^T q_\Phi(x_k|x_1,..,x_{k-1} )}]
    \end{equation*}
    \begin{equation*}
        D_{KL}(p_\theta || q_\Phi) = \mathbb{E}_{x \sim p_\theta} 
        \log[\frac{p_\theta(x_1)}{q_\Phi(x_1)}] + \mathbb{E}_{x \sim p_\theta} \sum_{k=2}^T \log(\frac{p_\theta(x_k|x_1,..,x_{k-1} )}{q_\Phi(x_k|x_1,..,x_{k-1} )}
    \end{equation*}
    \begin{equation*}
        D_{KL}(p_\theta || q_\Phi)=
        \mathbb{E}_{x_1 \sim p_\theta}  D_{KL} (p_\theta(x_1) || q_\Phi(x_1)) + 
        \sum_{k=2}^T \mathbb{E}_{x_1,...,x_{k-1}  \sim p_\theta} 
        D_{KL} (p_\theta(x_k| x_1,...,x_{k-1}) || q_\Phi(x_k| x_1,...,x_{k-1})
    \end{equation*}
    Now, by using the lemma we mentioned earlier we get( note that 
    $\mathbb{E}_{x_k \sim p_\theta} [x_k | x_1,...,x_{k-1}] = f_\theta(x_1,...,x_{k-1})$ and also $\mathbb{E}_{x_k \sim q_\Phi} [x_k | x_1,...,x_{k-1}] = g_\Phi(x_1,...,x_{k-1})$:
    \begin{equation*}
        D_{KL}(p_\theta || q_\Phi) = \mathbb{E}_{x_1,...,x_{k-1} \sim p_\theta} \frac{1}{2\sigma^2}\sum_{k=1}^T (f_\theta(x_1,...,x_{k-1}) - g_\Phi(x_1,...,x_{k-1}))^2
    \end{equation*}
    Now obviously, in order to generate a sample, we need to have all the prior samples generated due to the sequential dependencies.
    Furthermore, if $x$ is going to be a high dimensional random vector($T \rightarrow \infty)$, then getting $x_k | x_1, ..., x_k$ will be harder, since the neural network corresponding with $x_k$ given $x_1, ..., x_{k-1}$ will be much complex. Additionally, taking expectations will be harder as $k$ grows, since we need to integrate over joint distribution of $k$ variables.
    \\
    Now, instead of taking the expectation which is hard, we can use "Monte-Carlo" simulation. Since we know the distribution, we will generate a long sequence of data $\{x_k^i\}_{i=1}^M \sim p_\theta$ and  $\{y_k^i\}_{i=1}^M \sim q_\Phi$ and then calculating the square of their average difference as below:
    \begin{equation*}
        D_{KL}(p_\theta || q_\Phi) \approx \frac{1}{M} \sum_{m=1}^M  \frac{1}{2\sigma^2}\sum_{k=1}^T (f_\theta(x_1^m,...,x_{k-1}^m) - g_\Phi(x_1^m,...,x_{k-1}^m))^2
    \end{equation*}
    % --------------------------------------------------------------------------
    \item Well, the AR models cannot capture long-term dependencies for a couple of reasons. Firstly, we will run out of storage quickly as we go further and also computations will be more complex(even if we use methods as above such as Monte-Carlo distribution).\\
    Furthermore, "gradient vanishing" is another issue, as the network's depth increases, the sensitivity of the loss function with respect to the early steps will grow smaller and as a result it can't capture those long-term dependencies.
    \\
    To address this issue, we can use an RNN structure. Each RNN model maintains a hidden state that encapsulates past information. Rather than referencing the entire history of observations \( p(x_k | x_1, \ldots, x_{k-1}) \), it relies on a compressed version known as the hidden state \( p(x_k | h_{k-1}) \).
    % --------------------------------------------------------------------------
    \item The degradation of the model could be caused by the fact that the model's structure was not good so that it couldn't capture long-term dependencies. Furthermore, given the sequential structure of the model for generation of $x_k$, if an error ocurres in prior samples, it could propagate and accumulate and wreck havoc on the future samples.
\end{enumerate}
%-----------------------------------------------------------------------------
\section*{Problem 3}
 Since the $\pi_c$'s must sum to 1 over $c$'s, we can use the idea of the softmax to set them as($\theta = (W,b,v,u,d))$:
\begin{equation*}
    \pi_i^c = \frac{\exp(W_c^T h_i + b_i^c  )}{\sum_{k=1}^C \exp(W_k^T h_i + b_k^c  )}
\end{equation*}
and since $p(x_i | x_{1:i-1}; \theta) \sim \mathcal{N}(v_i^T h_i + b_i, e^{u_i^T h_i + d_i})$ and $p(x_i | x_{1:i-1}) \sim \sum_{c=1}^C \pi_i^c \mathcal{N}(\mu_i^c, \sigma_i^c)$, we will formulate $\mu_i^c$ and $\sigma_i^c$ as;
\begin{equation*}
    \mu_i^c = (v_i^c)^T h_i + b_i^c 
\end{equation*}
\begin{equation*}
    \sigma_i^c = \exp((u_i^c)^T h_i + d_i^c)
\end{equation*}
The number of parameters for $\pi$ is $C \times (D+1) $ since $\dim(\pi_i)=C$, $\dim(b_i)=C$ and $W \in \mathbb{R}^{C \times D}$. Likewise, the number of parameters of $\mu$ and $\sigma$ will be $C \times (D+1)$.
At last, the total number of required parameter is $3C \times (D+1)$. 
%-----------------------------------------------------------------------------
\section{Problem 4}
\begin{enumerate}
    \item if $x \sim  \mathcal{N}(0,2)$, then $\mathbb{E}[(x-0)^2] = 2$ and $\mathbb{E}[x] = 0$. Thus these two will suggest that:
    \begin{equation*}
        \mathbb{E}_{x \sim  \mathcal{N}(0,2)}[x^2 + x + 1] = 2 + 0 + 1 = 3
    \end{equation*}
    \item We know that for a function of a random variable $x$: 
    $\text{Var}(f(x)) = \mathbb{E} [f(x)^2] - (\mathbb{E} [f(x)])^2$
    So replacing the expectation with "Monte-Carlo" will result in:
    \begin{equation*}
        \text{Var(f(x))} \approx \frac{1}{M} \sum_{m=1}^M f(x_m)^2 - (\frac{1}{M}   \sum_{m=1}^M f(x_m))^2 
    \end{equation*}
    \item In order to achieve an unbiased estimator for $F(\theta)$, instead of using $N$ terms in the Monte-Carlo simulation, we can average on $K$ terms and Let also $\forall i \in {1,2,...,K} : m_i \sim Unif([1,2,..,N])$. \\
    So we have $F(\theta) = \sum_{n=1}^N w_n f(\theta; n) + \lambda R(\theta)$. As mentioned earlier, instead of using $N$ terms in the sum, we will use k terms and then scale the result by a factor of $\frac{N}{K}$ like:
    \begin{equation*}
        \hat{F}(\theta) = \frac{N}{K} \sum_{m_i \sim  Unif([1,2,..,N])}  w_{m_i} 
        f(\theta; m_i) + \lambda R(\theta)
    \end{equation*}
    Firstly, we can prove that this estimator is unbiased or in other words: 
    \begin{equation*}
    \mathbb{E}_{m_1 ,.., m_K} [\hat{F}(\theta)] = F(\theta)
    \end{equation*}
    which is trivial and can be obtained from:
    \begin{equation*}
        \mathbb{E}[ \hat{F}(\theta) ] = \mathbb{E} [\frac{N}{K} \sum_{m_i }  w_{m_i} f(\theta, m_i) + \lambda R(\theta) ] = \frac{N}{K} \mathbb{E} [ \sum_{m_i }  w_{m_i} f(\theta; m_i)] + \lambda R(\theta)
    \end{equation*}
    Now, we will assume that for each $w_k$, if they were selected,$y_k = w_k$, and otherwise they are zero. So we get that:
    \begin{equation*}
        \mathbb{E}[ \hat{F}(\theta) ] = \mathbb{E} [\frac{N}{K} \sum_{i=1}^N  y_{i} f(\theta, m_i) + \lambda R(\theta) ]
    \end{equation*} where $y_i $ is a random variable with binomial distribution. It is obvious that $\mathbb{P} [y_n \neq 0] = \frac{K}{N}$ and $\mathbb{P} [y_n =0] = \frac{N-K}{N}$. Therefore, it is easy to see that $\mathbb{E} [y_i]  = \frac{K}{N} w_i $ .
    \\
    So by linearity of expectation and the fact from above we will get that:
    \begin{equation*}
    \mathbb{E}[ \hat{F}(\theta) ] = \frac{N}{K} \sum_{i=1}^N \frac{K}{N} w_{i} f(\theta, i) + \lambda R(\theta) = F(\theta) 
    \end{equation*}
    We can do the same with gradient calculations. Now, assuming that the function call and the gradient are $o(1)$, if we sum up all $N$ terms we will get a complexity of $o(N)$ in calculating $F(\theta)$ and $ \nabla F(\theta)$.
    However, in the proposed method, the complexity of the calculation of $F(\theta)$ and $ \nabla F(\theta)$, will shrink to $o(K)$. \\
    However, we should note that $K$, can't be small, since we are replacing the expectation with the average.
\end{enumerate}


%-----------------------------------------------------------------------------
% \section*{Problem 5}
% %-----------------------------------------------------------------------------
% \begin{enumerate}
%     \item 
% \end{enumerate}
%-----------------------------------------------------------------------------
\end{document}
